%!TeX spellcheck = en_GB
%\blankpage
\chapter{Conclusion}
	Since the inception of BeSSeL and resultant accurate determination of Galactic structure as visible from the Northern Hemisphere, the astrometric community has strived to do the same in the Southern Hemisphere. Early attempts on the LBA were affected by suspected ionospheric effects, limited mutual bandwidth on the heterogeneous array and time unavailability. With the material contained within this thesis, it should now be possible to determine the structure and kinematics of the Galaxy as visible from Southern Hemisphere. This is a significant breakthrough.

\section{Summary of Results}	
	%In \hyperref[chap:chapter3]{Chapter~\ref*{chap:chapter3}}
	I reduce BeSSeL data and measure the proper motion and parallax for three 22~GHz water masers and a 6.7~GHz methanol maser. With these measurements I determine distances for the host star forming regions and use these distances to place all four masers in the Perseus arm of the Galaxy. Finally, I combine my results with previously known Perseus arm maser parallaxes to calculate a Perseus spiral arm pitch angle.
	
	%In \hyperref[chap:chapter4]{Chapter~\ref*{chap:chapter4}}
	I conduct a targeted VLBI survey of all known Southern Hemisphere 6.7~GHz methanol masers with flux density~$>10$~Jy. I model spatial and energetic properties for each maser velocity feature and determine overall maser compactness. I catalogue the individual maser properties and compactness for future studies and select out the best targets for astrometry.
	
	%In \hyperref[chap:chapter5]{Chapter~\ref*{chap:chapter5}} 
	I introduce and discuss inverse MultiView calibration. I derive relationships to predict how inverse MultiView will be able to remove residual delays. I find that inverse MultiView is robust to target source positional uncertainties, baseline, ionosphere and troposphere delay uncertainties given prior calibration. I expect uncertainty in inverse MultiView calibration to increase as $\theta^2$ rather than $\theta$ for traditional phase referencing.
	
	%\hyperref[chap:chapter6]{Chapter~\ref*{chap:chapter6}}
	Finally, I conduct pilot \spirals\space observations using quasars to test inverse MultiView and the BeSSeL observation/calibration approach as applicable to the new ASCI array. This process involves scheduling, observing, correlating, data reduction, inverse MultiView application and astrometric analysis. I find that inverse MultiView allows the ASCI array to achieve microacrsecond astrometry out to average target--calibrator separations $\sim7.5$~deg.
	
%\clearpage
\section{Current and Future work}
	\spirals\space began pilot observations of 6.7~GHz methanol masers in September 2019 and fully began taking observations in May 2020. At the time of writing, over 400~hours of data have been collected. All \spirals\space targets have been $A-$grade sources as catalogued. Early astrometric estimates show that inverse MultiView is working at the level of $40-60\mu$as per epoch, in--line with expectations.
	
	Before pilot observations begun, Warkworth Radio Astronomical Observatory, New Zealand joined the \spirals\space collaboration. They brought access to the Warkworth~30m radio telescope into ASCI. This extends the maximum baseline from Hobart--Katherine at $|B|=3500$~km to Yarragadee--Warkworth at $|B|=5500$~km and increases the array sensitivity. 
	
	Ceduna~30m, Yarragadee~12m and Warkworth~30m have a planned receiver upgrade. While all three upgrades are at various stages, all three are very likely to be completed in the next year. All receivers will have a mutual spanned bandwidth of at least 3--7 or 8~GHz. This spanned bandwidth will allow dispersive delay removal during geoblock fitting and consequential accurate zenith tropospheric delay determination. I presented evidence to suggest a large amount of residual zenith delay ($5$~cm) in inverse MultiView test observations from the phase slopes, so accurate determination would most likely leave only ionosphere residuals above 1~cm. As residual dry tropospheric delay has a tendency to diverge at a elevation dependant target--calibrator radius compared to the delay slopes expected from the much thinner ionosphere, this may allow either an extension to how low in elevation tracks can be or an observing scheme comprised of only one low--elevation track.
	
	%Accurate tropospheric calibration combined with inverse MultiView on quasars might allow to study structural changes of ionosphere -- Bayesian modelling of slope parameters. Probably requires more careful experimental design than these

	The parallax measured for the 22~GHz water maser G021.87$+$0.01 was not sufficiently constrained by the astrometry to directly determine a distance. This was largely due to observed source evolution and/or flux density variability common in water masers. It would be beneficial to independently confirm the distance of $D=13.7$~kpc as I determined from kinematic models of recession velocity and proper motion. The region in question appears to have a nearby 6.7~GHz methanol maser G021.880$+$0.014 \citep{Caswell1995a,Breen2015}. The methanol maser G021.880$+$0.014 is offset 0.6~amin from water maser G021.87$+$0.01 with identical velocity range $v=17$ to $22$\,\kms. However, the methanol maser has a catalogued flux density between $S=15$ and 5~Jy with unknown compactness. Therefore, it is not immediately obvious whether it is suitable for inverse phase referencing or inverse MultiView. 
	
	As I saw from the reduction of G021.87$+$0.01, only the nearby calibrator J1825--0737 at $\Delta\theta=2.566$~deg was suitable for phase referencing (at $\sim200$~mJy). The other calibrator J1835--1115 at $\Delta\theta=1.793$~deg was too--weak ($\sim20$~mJy). In the neighbourhood of $\Delta\theta<6$~deg there are 9 quasars suitable for phase referencing; all appear compact and have flux densities $S>80$~mJy. Depending on the compactness of the quasar, this could prove to be a very good chance to test non--inverse MultiView (by phase referencing on a quasar and interpolating phase slopes to maser) or inverse MultiView on the VLBA.
	
	I chose to conduct the MultiView tests at X--band, however, there is little preventing a repeat of observations at S--band ($\sim2$~GHz). All ASCI array telescopes (and the possible inclusion of Warkworth~12m) have access to S--band receivers, so there is the possibility to test whether MultiView holds up as residual delays increase on the target (due to the ionosphere). I attributed the measured phase slopes at X--band to residual dry troposphere, however, at S--band the ionosphere and variations to it should be dominant. I intend to pursue this in a future study.
	
	If dry tropospheric residuals, baseline offsets and ionospheric residuals were brought down to VLBA levels of 1~cm, for a target--calibrator separation of 3~deg I calculate the inverse MultiView should be able to achieve an astrometric accuracy of $\sigma_\theta=1\mu$as. This also implies that inverse MultiView on the VLBA should be approaching this astrometric accuracy already.
	
	The application of the high astrometric accuracy provided by MultiView carrying the most impact may be the ability to measure parallaxes to masers in the Large Magellanic Cloud \citep{Green2008,Green2009,Imai2013}. While this is not currently possible from a logistical perspective, it is no longer impossible. The ability to directly measure distances to star formation regions at $\sim 50$~kpc would affect the whole cosmic distance ladder by almost an order of magnitude; allowing the the whole Universe to be brought into sharper focus.
	
	Other very promising avenues are low--frequency astrometry, which struggle to surpass the limits of the ionosphere. Pulsar astrometry \citep{Deller2019} rely heavily on in--beam calibrators to achieve microsecond astrometry due to the overwhelming influence of the ionosphere at low frequencies (1.66~GHz). Even for target--calibrator separations 0.5~deg, which would be an in--beam calibrator on the VLBA, a residual TEC of only 1~TECU at 1.66~GHz would mean a delay difference of almost 15~cm! At 0.5~deg separation this constitutes an astrometric accuracy of $35\mu$as, however, this requires the existence of an in--beam calibrator. The ionosphere is largely planar at even lower frequencies \citep[$\le150$~MHz; ][ and given good weather]{Rioja2018} so MultiView should be able to achieve the $\theta^2$ dependence of plane fitting. With a similar setup at 1~deg separation, MultiView should achieve $1\mu$as astrometric accuracy.
	
	I have confidence in saying that the most significant result of this thesis is the demonstration of inverse MultiView calibration. Trigonometric parallax is the gold standard for distance estimates in astronomy and any ability to increase the accuracy that parallaxes can be determined affects all aspects of astronomy and cosmology. Inverse MultiView has the potential to be the next big leap in astrometric calibration that can allow accurate distance determination an order of magnitude larger.
	
	%\blankpage